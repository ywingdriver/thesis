%%%%%%%%%%%%%%%%%%%%%%%%%%%%%%%%%%%%%%%%%%%%%%%%%%%%%%%%%%%%%%%%%%%%%%%%%%%%%%%%%%%%%%%%%%%%%%%%%%%%%%%%%%%%%%%%%%%%%%%%%%%%%%%%%%%%%%%%%%%%%%%%%%%%%%%%%%%
% This is just an example/guide for you to refer to when submitting manuscripts to Frontiers, it is not mandatory to use Frontiers .cls files nor frontiers.tex  %
% This will only generate the Manuscript, the final article will be typeset by Frontiers after acceptance.   
%                                              %
%                                                                                                                                                         %
% When submitting your files, remember to upload this *tex file, the pdf generated with it, the *bib file (if bibliography is not within the *tex) and all the figures.
%%%%%%%%%%%%%%%%%%%%%%%%%%%%%%%%%%%%%%%%%%%%%%%%%%%%%%%%%%%%%%%%%%%%%%%%%%%%%%%%%%%%%%%%%%%%%%%%%%%%%%%%%%%%%%%%%%%%%%%%%%%%%%%%%%%%%%%%%%%%%%%%%%%%%%%%%%%

%%% Version 3.4 Generated 2018/06/15 %%%
%%% You will need to have the following packages installed: datetime, fmtcount, etoolbox, fcprefix, which are normally inlcuded in WinEdt. %%%
%%% In http://www.ctan.org/ you can find the packages and how to install them, if necessary. %%%
%%%  NB logo1.jpg is required in the path in order to correctly compile front page header %%%

\documentclass[utf8]{frontiersSCNS} % for Science, Engineering and Humanities and Social Sciences articles
%\documentclass[utf8]{frontiersHLTH} % for Health articles
%\documentclass[utf8]{frontiersFPHY} % for Physics and Applied Mathematics and Statistics articles

%\setcitestyle{square} % for Physics and Applied Mathematics and Statistics articles
\usepackage{url,hyperref,lineno,microtype,subcaption}
\usepackage[onehalfspacing]{setspace}

\linenumbers

\hypersetup{
    colorlinks,
    citecolor=blue,
    filecolor=black,
    linkcolor=blue,
    urlcolor=blue
}

% Leave a blank line between paragraphs instead of using \\


\def\keyFont{\fontsize{8}{11}\helveticabold }
\def\firstAuthorLast{Sample {et~al.}} %use et al only if is more than 1 author
\def\Authors{William Cook\,$^{1,*}$, Prof. Jesse Johnson\,$^{2}$ and Prof. Marko Maneta\,$^{1,2}$}
% Affiliations should be keyed to the author's name with superscript numbers and be listed as follows: Laboratory, Institute, Department, Organization, City, State abbreviation (USA, Canada, Australia), and Country (without detailed address information such as city zip codes or street names).
% If one of the authors has a change of address, list the new address below the correspondence details using a superscript symbol and use the same symbol to indicate the author in the author list.
\def\Address{$^{1}$Laboratory X, Institute X, Department X, Organization X, City X , State XX (only USA, Canada and Australia), Country X \\
$^{2}$Laboratory X, Institute X, Department X, Organization X, City X , State XX (only USA, Canada and Australia), Country X  }
% The Corresponding Author should be marked with an asterisk
% Provide the exact contact address (this time including street name and city zip code) and email of the corresponding author
\def\corrAuthor{Corresponding Author}

\def\corrEmail{william.cook@umontana.umt.edu}




\begin{document}
\onecolumn
\firstpage{1}

\title[Running Title]{Dual Ensemble Kalman Filter Applied To Raster Parameters in a Hydrologic Model} 

\author[\firstAuthorLast ]{\Authors} %This field will be automatically populated
\address{} %This field will be automatically populated
\correspondance{} %This field will be automatically populated

\extraAuth{}% If there are more than 1 corresponding author, comment this line and uncomment the next one.
%\extraAuth{corresponding Author2 \\ Laboratory X2, Institute X2, Department X2, Organization X2, Street X2, City X2 , State XX2 (only USA, Canada and Australia), Zip Code2, X2 Country X2, email2@uni2.edu}


\maketitle


\begin{abstract}

%%% Leave the Abstract empty if your article does not require one, please see the Summary Table for full details.
\section{}
Abstract goes here.


\tiny
 \keyFont{ \section{Keywords:} hydro, model, hydro, model, hydro, model, kalman, filter} %All article types: you may provide up to 8 keywords; at least 5 are mandatory.
\end{abstract}

\section{Introduction}

This research is part of the \textbf{daWUAP} project led by Marko Maneta. The \textbf{daWUAP} team has created a hydrological rainfall-runoff model that predicts streamflows across the state of Montana. \textbf{daWUAPhydroengine} is informed by a variety of smaller models, including a groundwater model, snow water equivalient (swe) model, and agricultural component model. Despite this complexity, \textbf{daWUAPhydroengine} is designed to be a quick and efficient model.

This paper focuses on the calibration of \textbf{daWUAPhydroengine} via a Dual State Parameter Estimation Ensemble Kalman Filter. Uniquely, this filter operates on multiple high dimensional parameters represented as raster images. These raster images are both ingested and outputted by the \textbf{daWUAPhydroengine} and were previously set at a constant value. The filter aims to both calibrate and distribute these variables geospatially. 

Hydrological models generate various states that are generally determined by a set of time-invariant parameters. In practice these parameters are calibrated continuously and are tweaked as soon as new observations become available. Parameter calibration techniques have recently been the subject of research \cite{Xie2010} \cite{Sorooshian1993}, although many of the earliest calibration methods are unable to account for all sources of error \cite{Evensen1994}. More recently, sequential parameter estimation through techniques such as the Ensemble Kalman Filter have been developed that can 1) simultaneously calibrate parameter and state estimates and 2) take all sources of uncertainty into account \cite{Evensen2003}.

The Kalman Filter is an efficient model for continuous parameter estimation because it is a sequential data assimilation algorithm, only needing the previous timestep's state estimate, parameter estimate, and co-variance matrices to update the current timestep's state estimate, parameter estimate, and co-variance matrices.  Although the original Kalman Filter \cite{Kalman1960} works well for linear problems, more advanced techniques must be used for non-linear dynamics. The extended Kalman Filter \cite{Jazwinski1970} works for mildly non-linear systems but does not function optimally on heavily non-linear systems \cite{Miller1994}. The Unscented Kalman Filter \cite{Julier1997} is an all-around improvement on the Extended Kalman Filter that allows for the filtering of highly non-linear systems. The Ensemble Kalman Filter \cite{Evensen1994}, a predecessor to the Unscented Kalman Filter, filters non-linear systems by generating an 'ensemble' of model instances and adding unique noise to each model's forcing data. The main advantage of this ensemble based approach is the substitution of the original Kalman Filter's error covarience matrix with an ensemble covariance matrix, which allows for the efficient computation of the covariance of high dimensional state vectors. The Dual State Ensemble Kalman Filter \cite{Moradkhani2005} extends the Ensemble Kalman Filter to also include static parameter estimation. This paper examines the Dual State Ensemble Kalman Filter's application to high dimensional geospatially distributed raster data.

Research has been done on applying dual state parameter estimation ensemble Kalman filtering to hydrologic models \cite{Moradkhani2005}. Further research into the calibration of hydrologic parameters, particularly high dimensional geospatial parameters, will help both inform future hydrologic models and allow for examination of the effectiveness of different techniques that distribute geospatial parameters across a landscape.

The optimization of the parameters must be done in a way that is reasonably efficient. Since models such as \textbf{daWUAPhydroengine} ingest and output dense raster data, for example, many simpler calibration implementations will become unwieldy because of large co-variance matrices. Furthermore, since \textbf{daWUAPhydroengine} is designed to be a quick and efficient model, it is preferable that any calibration component is similarly efficient.

The main contribution of this paper is a new 'moving window' model for kernel smoothing. The 'moving window' technique allows the parameter perturbation technique be informed by a set number of previous timesteps. This paper also covers the case application of the Dual Ensemble Kalman Filter framework to the \textbf{daWUAP} hydrologic engine, which necessitates the calibration of 4 large raster parameters. Finally, the code used for this paper is modular and should be usable with any Bayesian model.

Section 2 covers the theory behind the Dual State Ensemble Kalman Filter. Section 3 discusses the 'moving window' improvement and its implementation within the DSKE parameter sampling methods. Section 4 covers the application of the Dual State Ensemble Kalman Filter to \textbf{daWUAPhydroengine}.


% For Original Research articles, please note that the Material and Methods section can be placed in any of the following ways: before Results, before Discussion or after Discussion.

\section{The Dual Ensemble Kalman Filter}

According to Jazwinski \cite{Jazwinski1970} any discrete nonlinear stochastic-dynamic model can be defined as:

\begin{equation}\label{eq:gen_stoc}
x_{t+1} = f(x_{t}, u_{t}, \theta_{t}) + \varepsilon_{t}
\end{equation}

where $x_{t}$ is an $n$ dimensional vector representing the state variables of the model at time step $t$, $u_{t}$ is a vector of forcing data (e.g temperature or precipitation) at time step $t$, and $\theta_{t}$ is a vector of model parameters which may or may not change per time step (e.g \textit{soil beta }or \textit{DDF}). The non-linear function $f$ takes these variables as inputs. The noise variable $\varepsilon_{t}$ accounts for both model structural error and for any uncertainty in the forcing data.

A state's observation vector $z_{t}$ can be defined as

\begin{equation}\label{eq:gen_obs}
z_{t} = h(x_{t}, \theta_{t}) + \delta_{t}
\end{equation}

Where the $x_{t}$ vector represents the true state, $\theta_{t}$ represents the true parameters, $h(.)$ is a function that determines the relationship between observation and state vectors, and $\delta_{t}$ represents observation error. $\delta_{t}$ is Gaussian and independent of $\varepsilon_{t}$.

The Dual State Ensemble Kalman Filter can be split into three subsections: The prediction phase, the parameter correction phase, and the state correction phase. 

\subsection{Prediction Phase}

In a Dual Ensemble Kalman filter, each ensemble member \textit{i} is represented by a stochastic model similar to \eqref{eq:gen_stoc}. The modified equation is as follows:

\begin{equation}\label{eq:dekf_predict}
x_{t+1}^{i-} = f(x_{t}^{i+}, u_{t}^{i}, \theta^{i-}_{t}) + \omega_{t}, \quad i=1,...,n
\end{equation}

Where $n$ is the total number of ensembles. The $-/+$ superscripts denote corrected ($+$) and uncorrected ($-$) values. Note that $\theta^{i-}_{t}$'s $t$ superscript does not necessarily denote that $\theta$ is time variant but rather indicates that parameter values change as they are filtered over time. The noise term $\omega_{t}$ accounts for model error and will hereafter be excluded from the state equation.

Errors in the forcing data are accounted for through the perturbation the forcing data vector $u_{t}$ with random noise $\zeta_{t}^{i}$ to generate a unique variable $u_{t}^{i}$ for each ensemble. $\zeta_{t}^{i}$ is drawn from a normal distribution with a covarience matrix $Q_{t}^{i}$.

\begin{equation}\label{eq:dekf_u}
u_{t+1}^{i} = u_{t} + \zeta_{t}^{i}, \quad \zeta_{t}^{i} \sim N(0,Q_{t}^{i}) 
\end{equation}

To generate the priori parameters $\theta^{i-}_{t+1}$ an evolution of the parameters similar to the evolution of the state variables must be implemented. To accomplish this the kernel smoothing technique developed by West\cite{West1993} is used. Legacy implementations of parameter evolution added a small perturbation sampled from $N(0,\Sigma^{\theta}_{t})$, where $\Sigma^{\theta}_{t}$ represents the covariance matrix of $\theta$ at timestep $t$. This legacy method of evolution resulted in overly disposed parameter samples and the loss of continuity between two consecutive points in time\cite{Chen2008}. Kernel smoothing has been used effectively to solve this problem in previous Dual Ensemble Kalman filter implementations \cite{Moradkhani2005} and similar models \cite{Chen2008}.

\begin{equation}\label{eq:dekf_thetaminus}
\theta_{t+1}^{i-} = a\theta_{t}^{i+} + (1-a)\bar{\theta}_{t}^{+} + \tau_{t}^{i}
\end{equation}
\begin{equation}\label{eq:dekf_tau}
\tau_{t}^{i} = N(0, h^{2}V_{t})
\end{equation}
 
Where $\bar{\theta}_{t}^{+}$ is the mean of the parameters with respect to the ensembles, $V_{t} = var(\theta_{t}^{i+})$, $a$ is a shrinkage factor between (0,1) of the kernel location, and $h$ is a smoothing factor. $h$ is defined by $\sqrt{1-a1/2}$, while $a$ is generally between (.45,.49). Note that $h$ and $a$ tend to vary per model and optimal values for these parameters are generally found via experimentation  \cite{Moradkhani2005}  \cite{Anderson1999} \cite{Annan2005} \cite{Chen2008}.

\subsection{Parameter Correction Phase}

In an Ensemble Kalman Filter, observations are perturbed to reflect model error. Therefore, the variable $z_{t+1}^{i}$ is defined as follows:

\begin{equation}\label{eq:dekf_obs}
z_{t+1}^{i} = z_{t+1} + \eta_{t+1}^{i},\quad \eta_{t+1}^{i} = N(0,R_{t+1})
\end{equation}

Where $z_{t+1}$ is an observation vector defined by \eqref{eq:gen_obs} and $\eta_{t+1}^{i}$ is a random perturbation drawn from a normal distribution with covarience matrix $R_{t+1}$. A set of state predictions that can be related to the observations are generated by running the priori state vector through the function $h(.)$:

\begin{equation}\label{eq:dekf_pred}
\hat{y}_{t+1}^{i} = h(x_{t+1}^{i-}, \theta_{t+1}^{i-})
\end{equation}

The parameter update equation is similar to the update equation of the linear Kalman filter ($\hat{x}^{+}_{t} = \hat{x}^{-}_{t} + K_{t}(z_{t}-H\hat{x}_{t})$) Notably,  parameters are corrected in lieu of the states:

\begin{equation}\label{eq:dekf_param_update}
\theta_{t+1}^{i+} = \theta_{t+1}^{i-} + K_{t+1}^{\theta}(z_{t+1}^{i}-\hat{y}_{t+1}^{i})
\end{equation}

To facilitate this, $K_{t+1}^{\theta}$ is defined as

\begin{equation}\label{eq:dekf_param_k}
K_{t+1}^{\theta} = \frac{\Sigma^{\theta,\hat{y}}_{t+1}}{\Sigma^{\hat{y},\hat{y}}_{t+1} + R_{t+1}}
\end{equation}

where $\Sigma^{\theta,\hat{y}}_{t+1}$ is the cross covariance of $\theta_{t+1}$ and $\hat{y}_{t+1}$, $\Sigma^{\hat{y},\hat{y}}_{t+1}$ is the covarience of $\hat{y}_{t+1}$, and $R_{t+1}$ is the observation error matrix from \eqref{eq:dekf_obs}. 

\subsection{State Correction Phase}

After $\theta_{t+1}^{i+}$ has been calculated the model is run again \eqref{eq:dekf_predict} with the $\theta_{t+1}^{i+}$ replacing $\theta_{t+1}^{i-}$.

\begin{equation}\label{eq:dekf_predict_2}
x_{t+1}^{i-} = f(x_{t}^{i+}, u_{t}^{i}, \theta^{i+}_{t}), \quad i=1,...,n
\end{equation}

After a new state vector is generated it is re-run through \eqref{eq:dekf_pred} with the new parameter vector:

\begin{equation}\label{eq:dekf_pred_2}
\hat{y}_{t+1}^{i} = h(x_{t+1}^{i-}, \theta_{t+1}^{i+})
\end{equation}

The corrected state vector is then run through the state update equation

\begin{equation}\label{eq:dekf_state_update}
x_{t+1}^{i+} = x_{t+1}^{i-} + K_{t+1}^{x}(z_{t+1}^{i}-\hat{y}_{t+1}^{i})
\end{equation}
 
\begin{equation}\label{eq:dekf_param_k}
K_{t+1}^{x} = \frac{\Sigma^{x,\hat{y}}_{t+1}}{\Sigma^{\hat{y},\hat{y}}_{t+1} + R_{t+1}}
\end{equation}

where $\Sigma^{x,\hat{y}}_{t+1}$ is the cross covariance of $x_{t+1}$ and $\hat{y}_{t+1}$.




\bibliographystyle{frontiersinSCNS_ENG_HUMS} % for Science, Engineering and Humanities and Social Sciences articles, for Humanities and Social Sciences articles please include page numbers in the in-text citations
%\bibliographystyle{frontiersinHLTH&FPHY} % for Health, Physics and Mathematics articles
\bibliography{test}

%%% Make sure to upload the bib file along with the tex file and PDF
%%% Please see the test.bib file for some examples of references

\section*{Figure captions}

%%% Please be aware that for original research articles we only permit a combined number of 15 figures and tables, one figure with multiple subfigures will count as only one figure.
%%% Use this if adding the figures directly in the mansucript, if so, please remember to also upload the files when submitting your article
%%% There is no need for adding the file termination, as long as you indicate where the file is saved. In the examples below the files (logo1.eps and logos.eps) are in the Frontiers LaTeX folder
%%% If using *.tif files convert them to .jpg or .png
%%%  NB logo1.eps is required in the path in order to correctly compile front page header %%%

\begin{figure}[h!]
\begin{center}
\includegraphics[width=10cm]{logo1}% This is a *.eps file
\end{center}
\caption{ Enter the caption for your figure here.  Repeat as  necessary for each of your figures}\label{fig:1}
\end{figure}


\begin{figure}[h!]
\begin{center}
\includegraphics[width=15cm]{logos}
\end{center}
\caption{This is a figure with sub figures, \textbf{(A)} is one logo, \textbf{(B)} is a different logo.}\label{fig:2}
\end{figure}

%%% If you are submitting a figure with subfigures please combine these into one image file with part labels integrated.
%%% If you don't add the figures in the LaTeX files, please upload them when submitting the article.
%%% Frontiers will add the figures at the end of the provisional pdf automatically
%%% The use of LaTeX coding to draw Diagrams/Figures/Structures should be avoided. They should be external callouts including graphics.

\end{document}
