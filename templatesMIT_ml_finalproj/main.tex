% -*- Mode:TeX -*-

%% IMPORTANT: The official thesis specifications are available at:
%%            http://libraries.mit.edu/archives/thesis-specs/
%%
%%            Please verify your thesis' formatting and copyright
%%            assignment before submission.  If you notice any
%%            discrepancies between these templates and the 
%%            MIT Libraries' specs, please let us know
%%            by e-mailing thesis@mit.edu

%% The documentclass options along with the pagestyle can be used to generate
%% a technical report, a draft copy, or a regular thesis.  You may need to
%% re-specify the pagestyle after you \include  cover.tex.  For more
%% information, see the first few lines of mitthesis.cls. 

%\documentclass[12pt,vi,twoside]{mitthesis}
%%
%%  If you want your thesis copyright to you instead of MIT, use the
%%  ``vi'' option, as above.
%%
%\documentclass[12pt,twoside,leftblank]{mitthesis}
%%
%% If you want blank pages before new chapters to be labelled ``This
%% Page Intentionally Left Blank'', use the ``leftblank'' option, as
%% above. 

\documentclass[12pt,twoside]{mitthesis}
\usepackage{lgrind}
%% These have been added at the request of the MIT Libraries, because
%% some PDF conversions mess up the ligatures.  -LB, 1/22/2014
\usepackage{cmap}
\usepackage{amsmath}
\usepackage[T1]{fontenc}
\usepackage{graphicx}
\usepackage{subcaption}
\graphicspath{ {./images/} }
% Added by me - makes the TOC into hyperlinks
\usepackage{color}
\usepackage{hyperref}
\hypersetup{
    colorlinks,
    citecolor=blue,
    filecolor=black,
    linkcolor=blue,
    urlcolor=blue
}

\pagestyle{plain}


%% This bit allows you to either specify only the files which you wish to
%% process, or `all' to process all files which you \include.
%% Krishna Sethuraman (1990).

\begin{document}
\pagestyle{plain}

\begin{flushright}
William Cook\\
Machine Learning\\
Final Project\\
Due 12/14/2018\\
\end{flushright}

\section{Abstract}

The \textbf{daWUAP} team at the University of Montana has created a hydrological rainfall-runoff model that predicts streamflows across the state of Montana. \textbf{daWUAPhydroengine} is informed by a variety of smaller models, including a groundwater model, snow water equivalient (swe) model, and agricultural component model. Despite this complexity, \textbf{daWUAPhydroengine} is designed to be a quick and efficient model.

This paper focuses on the calibration of \textbf{daWUAPhydroengine} via a Dual State Parameter Estimation Ensemble Kalman Filter. Uniquely, this filter operates on multiple high dimensional parameters represented as raster images. These raster images are both ingested and outputted by the \textbf{daWUAPhydroengine} and were previously set at a constant value. The filter aims to both calibrate and distribute these variables geospatially. 

\include{contents}
\chapter{Introduction}



	Utilizing sequential data assimilation techniques to filter hydrologic models is an efficient way to correct and calibrate them both before and after implementation in the field. Many observations such as SWE, streamflow, and precipitation are collected on a daily basis across various geographic regions, allowing the previous day's information to be dynamically ingested by the hydrologic model and inform current predictions. These models allow hydrologists to understand the past and predict the future.
	
	Models that ingest data sequentially can be efficiently corrected by a Kalman Filter, a sequential data assimilation algorithm. Kalman Filters only need the previous timestep's state estimate, parameter estimate, and co-variance matrices to update the current timestep's state estimate, parameter estimate, and co-variance matrices. However, the original Kalman filter\cite{Kalman1960} was created to solve linear problems, and more complicated implementations must be used to solve non-linear problems. The extended Kalman Filter\cite{Jazwinski1970} works for mildly non-linear systems but does not function optimally on heavily non-linear systems\cite{Miller1994}. The Unscented Kalman Filter\cite{Julier1997} is an all-around improvement on the Extended Kalman Filter that allows for the filtering of highly non-linear systems. The Ensemble Kalman Filter\cite{Evensen1994}, a predecessor to the Unscented Kalman Filter, filters non-linear systems by generating an 'ensemble' of model instances and adding unique noise to each model's forcing data. The main advantage of this ensemble based approach is the substitution of the original Kalman Filter's error covarience matrix with an ensemble covariance matrix, which allows for the efficient computation of the covariance of high dimensional state vectors. 
	
	In order to calibrate parameters within a hydrologic model a Dual State Kalman Filter may be used. Dual state Kalman filters add a small perturbation to a series of parameters that the use rwishes to calibrate. These perturbed parameters vectors are then corrected in a similar fashion to the state vectors. After this happens a 'second' filter runs and corrects the state vectors in the normal fashion. The Dual State Ensemble Kalman Filter implemented by Moradkhani et. al\cite{Moradkhani2005} extends the Ensemble Kalman Filter into a dual state configuration.
	
	 To examines the Dual State Ensemble Kalman Filter's application to high dimensional geospatially distributed raster data Professor Marko Maneta's \textbf{daWUAPhydroengine} is used. Professor Maneta and his team have created a hydrological rainfall-runoff model that predicts streamflows across the state of Montana. \textbf{daWUAPhydroengine} is informed by a variety of smaller models, including a groundwater model, snow water equivalient (swe) model, and agricultural component model. Despite this complexity, \textbf{daWUAPhydroengine} is designed to be a quick and efficient model. \textbf{daWUAPhydroengine} is influenced by a series of uncalibrated high-dimensional parameters that are stored as 2D raster data. A Dual Ensemble Kalman Filter is an optimal calibration algorithm to calibrate this raster data because 1) the DEnKF does not have to compute the high dimensional state covariance matrix and 2) \textbf{daWUAPhydroengine} is a sequential model.
	
	Chapter 2 covers the methods behind the Dual Ensemble Kalman Filtering algorithm originally implemented by Moradkhani et al. Chapter 3 displays results of applying the Dual State Kalman Filter to \textbf{daWUAPhydroengine}. Chapter 4 discusses improvements currently being made, particularly the plans for a new, innovative hierarchical algorithm.
	
	
	

	
%% This is an example first chapter.  You should put chapter/appendix that you
%% write into a separate file, and add a line \include{yourfilename} to
%% main.tex, where `yourfilename.tex' is the name of the chapter/appendix file.
%% You can process specific files by typing their names in at the 
%% \files=
%% prompt when you run the file main.tex through LaTeX.
\chapter{Theory}



\section{The History of Kalman Filters}

R.E Kalman published the article \textit{A New Approach to Linear Filtering and Prediction Problems} in 1960 \cite{Kalman1960}. Since then, the so-called "Kalman Filter" has been tested, researched, and improved extensively. Kalman's original algorithm was limited to linear systems. The development of the Extended Kalman Filter allowed Kalman Filters to operate on non-linear systems with some limitations. More recently, the Unscented Kalman Filter \cite{Julier1997} and the Ensemble Kalman Filter \cite{Evensen1994} have been developed to work on non-linear systems.

\section{The Linear Kalman Filter}

The original Kalman filter was created to solve problems where both a predictive sequential model and a series of observations is available. The predictive model can be represented as the linear stochastic difference equation
\begin{large}
\begin{equation}\label{eq:2p1}
x_{i} = Ax_{i-1} + Bu_{i-1} + w_{i-1}
\end{equation}
\end{large}

Where $A$ is the model matrix which serves to transform the vector $x_{i-1}$ to the current timestep, $B$ is the control matrix that transforms the control vector $u_{i}$ to account for external forces on the model, $w_{i}$ is a vector of model error, and $i$ is the timestep.

An observation for any given timestep $i$ can be represented as 

\begin{large}
\begin{equation}\label{eq:2p2}
z_{i} = Hx_{i} + v_{i}
\end{equation}
\end{large}

where $z_{i}$ is the vector of observations, $x_{i}$ is the vector of true states, $H$ is a masking matrix, and $v_{i}$ is a vector of measurement errors. $w_{i}$ and $v_{i}$ are assumed to be independent, normally distributed random variables with probability distributions defined by

\begin{large}
\begin{equation}\label{eq:2p3}
P(w) \sim N(0,Q)
\end{equation}
\begin{equation}\label{eq:2p4}
P(v) \sim N(0,R)
\end{equation}
\end{large}

\subsection{Algorithm}

Kalman filters optimize model predictions by blending predicted states with that timestep's observations. Conveniently, the algorithm's steps are separated into \textit{prediction} and \textit{update} categories. The initial prediction algorithm \eqref{eq:2p5} obtains the current timestep's vector of states using the same equation as \eqref{eq:2p1} with the removal of the random unknown vector $w$. To track the effects of ignoring $w$ the prior error covariance matrix $P^{-}$ is calculated \eqref{eq:2p6}.

\begin{table}[h]
\caption{Prediction Equations - Discrete Kalman Filter} 
\centering
\begin{tabular}{c c}
\\ [0.1ex] 
\hline   
Name & Equation \\
\hline
Model Prediction & \parbox{3cm}{\begin{equation}\label{eq:2p5} \hat{x}^{-}_{i} = A\hat{x}^{+}_{i-1} + Bu_{i-1} \end{equation}} \\
Update Prior Covariance & \parbox{3cm}{\begin{equation}\label{eq:2p6} P^{-}_{i} = AP^{+}_{i}A^{T}+Q \end{equation}}
\end{tabular}
\label{tab:hresult}
\end{table}

Equation \eqref{eq:2p8} returns the updated prediction $\hat{x}^{+}_{i}$ by multiplying the innovation between the observation and the masked prediction by the kalman gain $K$, which is defined in \eqref{eq:2p7}. Finally, the error covariance matrix is updated in \eqref{eq:2p9} to reflect the more accurate nature of the updated prediction.

\begin{table}[h]
\caption{Update Equations - Discrete Kalman Filter} 
\centering
\begin{tabular}{c c}
\\ [0.1ex]
\hline
Name & Equation \\ [0.5ex]
\hline            
Kalman Gain & \parbox{3cm}{\begin{equation}\label{eq:2p7}K_{i} = P^{-}_{i}H^{T}(HP^{-}_{i}H^{T} + R)^{-1} \end{equation}} \\
Update Estimate & \parbox{3cm}{\begin{equation}\label{eq:2p8} \hat{x}^{+}_{i} = \hat{x}^{-}_{i} + K_{i}(z_{i}-H\hat{x}_{i}) \end{equation}} \\
Update Posterior Covariance & \parbox{3cm}{\begin{equation}\label{eq:2p9}P^{+}_{i} = (I-K_{i}H)P^{-}_{i} \end{equation}}
\end{tabular}
\label{tab:hresult}
\end{table}


\section{The Whatever We call It Ensemble Kalman Filter}

The Dual Ensemble Kalman Filter is a sequential data assimilation method. It can be split into four different subsections: The initial prediction phase, the parameter correction phase, the second prediction phase, and the state correction phase.

\subsection{The Prediction Phase}

According to Jazwinski \cite{Jazwinski1970} any discrete nonlinear stochastic model can be defined as:

\begin{equation}\label{eq:gen_stoc}
x_{t+1} = f(x_{t}, u_{t}, \theta_{t}) + \omega_{t}
\end{equation}

where $x_{t}$ is an $n$ dimensional vector representing the state variables of the model at time step $t$, $u_{t}$ is a vector of forcing data (e.g temperature or precipitation) at time step $t$, and $\theta_{t}$ is a vector of model parameters which may or may not change per time step (e.g soil beta or DDF). The non-linear function $f$ takes these variables as inputs. The error variable $\omega_{t}$ accounts for both model structural error and for any uncertainty in the forcing data. 




%% This is an example first chapter.  You should put chapter/appendix that you
%% write into a separate file, and add a line \include{yourfilename} to
%% main.tex, where `yourfilename.tex' is the name of the chapter/appendix file.
%% You can process specific files by typing their names in at the 
%% \files=
%% prompt when you run the file main.tex through LaTeX.
\chapter{Application of DEnHKF to Hydrologic Model}

\section{daWUAPhydroengine}

The \textbf{daWUAPhydroengine} hydrologic dynamic model is used to test the viability of the DEnHKF method.  \textbf{daWUAPhydroengine} takes streamflow and subbasin parameters, precipitation, minimum temperatures, and maximum temperatures as inputs and outputs streamflow data along with some additional states such as snow water equivalent. \textbf{daWUAPhydroengine} was designed to be implemented in any geographic location. For this study it was utilized to model streamflows throughout the state of Montana.

\begin{table}[]
\caption{States} 
\begin{tabular}{lll}
State ($x$) & Purpose                              & Dimensions  \\ \hline
streamflow  & Streamflow (in cumecs)               & 330   \\
swe         & Snow Water Equivalent  (in $mm^{3}$) & 45012
\end{tabular}
\label{tab:states}
\end{table}

Configuring \textbf{daWUAPhydroengine} to model streamflows throughout Montana is advantageous because it allows for the calibration of a very large number of spatially distributed, high dimensional parameters. These parameters span the entirety of Montana, which covers an area of 380,800 $km^{2}$. Montana's large geographical coverage is diverse and the terrain differs in various ways (soil composition, forestation, etc.)

\begin{table}[]
\caption{Forcing Data} 
\begin{tabular}{lll}
Forcing Data ($u$) & Purpose                          & Dimensions \\ \hline
tempmin          & Lowest temperature for timestep  & 45012 \\
tempmax          & Highest temperature for timestep & 45012 \\
precipitation      & Amount of rainfall for timestep & 45012 
\end{tabular}
\label{tab:u_params}
\end{table}

\begin{table}[]
\caption{Calibrated Parameters} 
\begin{tabular}{lll}
Parameter ($\theta$) & Purpose                                                    & Dimensions  \\ \hline
ddf                  & Controls Rate of Snowfall                                        & 45012 \\
aet\_lp              & Controls AET                                                      & 45012 \\
soil\_beta           & Controls portion of ponded water that goes into soil storage & 45012 \\
soil\_max\_wat       & Controls soil maximum water capacity & 45012
\end{tabular}
\label{tab:t_params}
\end{table}

\begin{figure}[h]
    \centering
    \includegraphics[width=0.95\textwidth]{elevation}
    \caption{Elevation throughout Montana}
    \label{fig:elevation}
\end{figure}

\section{Observation Data}

A Kalman Filter relies on one or more observed states for correction. Accordingly, observations were obtained for streamflows across Montana and snowfall across Montana. For streamflow, USGS streamflow data was collected at 86 sites. Each observed site was paired with the closest simulated \textbf{daWUAPhydroengine} stream outlet within a 2.5 mile cutoff. For snowfall, SNOWTEL sites monitored by the Natural Resources Conservation Service (NRCS) were used. 90 stations were chosen and matched to specific pixels in \textbf{daWUAPhydroengine}'s raster files.


\begin{table}[]
\caption{Observations} 
\begin{tabular}{lll}
Observed State ($x$) & Source                              & Dimensions  \\ \hline
streamflow  & USGS & 82   \\
swe         & NRCS & 90
\end{tabular}
\label{tab:obs}
\end{table}

\begin{figure}[h]
    \centering
    \includegraphics[width=0.95\textwidth]{stations}
    \caption{all SWE stations plotted against modeled streamflows}
    \label{fig:stations}
\end{figure}
%% This is an example first chapter.  You should put chapter/appendix that you
%% write into a separate file, and add a line \include{yourfilename} to
%% main.tex, where `yourfilename.tex' is the name of the chapter/appendix file.
%% You can process specific files by typing their names in at the 
%% \files=
%% prompt when you run the file main.tex through LaTeX.
\chapter{Results}

\section{Perturation of States}

Early attempts at running the DSHKEnKF on the hydrologic model were marked by the complete collapse of posterior ensemble covariance to the mean and erratic jumps from the minimum to the maximum bounds for all streamflow parameters. Snow water equivalent parameters and states, however, converged in a stable fashion. It was determined that these erratic jumps were due to the hydrologic model's dependence on the value of the catchments' lowest groundwater reservoir, an unobserved and uncorrected state, which was integral to the production of streamflow in each timestep. White noise added to the forcing data (precipitation and min/max temperature) was unable to generate adequately diverse ensemble behavior when groundwater states were uniform across ensembles. To account for this, perturbation of groundwater and streamflow states was implemented.

\subsection{Perturbation of Groundwater States}

The hydrologic model was extremely sensitive to its starting states, in particular the lower groundwater reservoir. A low groundwater would lead to underwhelming groundwater, incentivising the parameters \textit{ck0}, \textit{ck1}, and \textit{ck2} to converge towards values that emptied all water pouring into the reservoirs so modeled streamflow could match the observations. Conversely, high starting groundwater caused \textit{ck0}, \textit{ck1}, and \textit{ck2} to converge towards parameters that let very little groundwater out of the reservoirs, further exasperating the problem and causing higher and higher values for \textit{ck0}, \textit{ck1}, and \textit{ck2} to be chosen. 

To solve this issue and find a reliable blanket starting value for groundwater subcatchments in an efficient amount of time the small dataset was utilized alongside the parameter boundaries specified by \cite{Maneta2008}. Trial and error was utilized on the dataset until groundwater stabilized. To encourage the model to explore different parameter values for different amounts of groundwater, initial states for the large dataset were perturbed across all ensembles and catchments using a $\mu$ equal to the average stable value of the small dataset, which for this model was roughly calculated to be 100mm, and a $\sigma$ of 80mm. During the prediction phase groundwater was treated as forcing data and was perturbed slightly at a $\sigma$ of $u_{gw} * gw$, with $u_{gw}$  at .05%.

\begin{figure}
\centering
\begin{minipage}{.5\textwidth}
  \centering
  \includegraphics[width=.98\linewidth]{bad_gw}
  \captionof{figure}{Uniform groundwater}
  \label{fig:bad_gw}
\end{minipage}%
\begin{minipage}{.5\textwidth}
  \centering
  \includegraphics[width=.94\linewidth]{good_gw}
  \captionof{figure}{Perturbed groundwater}
  \label{fig:good_gw}
\end{minipage}
\end{figure}


\subsection{Continuous perturbation of streamflow and swe states}

Another method of decoupling the model's calibration process from its over-reliance on groundwater was through the direct perturbation of streamflow and swe states. This perturbation guaranteed that ensemble collapse was never fully realized.


\section{Small dataset}

The small dataset, comprised of 3 catchments around the Biterroot valley, converged to a series of parameters very quickly. The small dataset was run over a period of 1800 days, or a little under 5 years. The simulation began in September so modeled snowfall accumulation could be corrected first, allowing accurate snowmelts to inform streamflow runoff in the Spring and Summer. 


\begin{table}[]
\caption{Hyperparameters - parameter perturbations and min/max ranges} 
\begin{tabular}{llll}
Parameter ($\theta$) & $q$ & Min & Max \\ \hline
Degree Day Factor (\textit{ddf})                 & .75mm$^\circ$C$^{-1}$d$^{-1}$ & 1mm$^\circ$C$^{-1}$d$^{-1}$ & 8mm$^\circ$C$^{-1}$d$^{-1}$ \\
Tempature Threshold (\textit{thres})                & .5$^\circ$C & -2.5$^\circ$C & 2.5$^\circ$C \\
Potential Evapo-Transpiration (\textit{aet\_lp})              & .15 & .3 & 1\\
Ponded water to soil storage (\textit{soil\_beta})          & 1.75 & 1 & 6 \\
Soil compartment max capacity (\textit{soil\_max\_wat})       & 40 & 50mm & 500mm \\
Immediate runoff (\textit{ck0})       & 6d$^{-1}$ & .25d$^{-1}$ & 10d$^{-1}$ \\
Fast runoff (\textit{ck1})      & 25d$^{-1}$ & 3.33d$^{-1}$ & 50d$^{-1}$\\
Groundwater runoff (\textit{ck2})       & 350d$^{-1}$ & 50d$^{-1}$ & 650d$^{-1}$ \\
Groundwater water storage threshold (\textit{hl1})       & 25mm & 0mm & 50mm \\
Groundwater peculation (\textit{perc})       & 1.5d & 3d & 50d \\
Wave celerity (\textit{K})         & 82400 & 81576 & 84872 \\
Wave dispersion (\textit{e})       & .35 & .25 & .4 \\
\end{tabular}
\label{tab:t_param_min_max}
\end{table}


\appendix
\include{biblio}
\end{document}

