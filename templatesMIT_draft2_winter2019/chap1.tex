\chapter{Introduction}



	Utilizing sequential data assimilation techniques to filter hydrologic models is an efficient way to correct and calibrate them both before and after implementation in the field. Many observations such as SWE, streamflow, and precipitation are collected on a daily basis across various geographic regions, allowing the previous day's information to be dynamically ingested by the hydrologic model and inform present and future predictions. More accurate models allow hydrologists to better understand the past and predict the future.
	
	Models that ingest data sequentially can be efficiently corrected by a Kalman Filter, a sequential data assimilation algorithm. Kalman Filters only need the previous timestep's state estimate, parameter estimate, and co-variance matrices to update the current timestep's state estimate, parameter estimate, and co-variance matrices. The original Kalman filter\cite{Kalman1960} was created to solve linear problems and more complicated implementations must be used to solve non-linear problems. The extended Kalman Filter\cite{Jazwinski1970} works for mildly non-linear systems but does not function optimally on heavily non-linear systems\cite{Miller1994}. The Unscented Kalman Filter\cite{Julier1997} is an all-around improvement on the Extended Kalman Filter that allows for the filtering of highly non-linear systems. The Ensemble Kalman Filter\cite{Evensen1994}, a predecessor to the Unscented Kalman Filter, filters non-linear systems by generating an 'ensemble' of model instances and adding unique noise to each model's forcing data. The main advantage of this ensemble based approach is the substitution of the original Kalman Filter's error covarience matrix with an ensemble covariance matrix, which allows for the efficient computation of the covariance of high dimensional state vectors. 
	
	In order to calibrate parameters within a hydrologic model a Dual State Kalman Filter may be used as demonstrated by Moradkhani et. al in 2005 \cite{Moradkhani2005}. Dual state Kalman filters add a small perturbation to a series of parameters that the userwishes to calibrate. These perturbed parameters vectors are then corrected in a similar fashion to the state vectors. After this happens a 'second' filter runs and corrects the state vectors in the normal fashion. The Dual State Ensemble Kalman Filter implemented by Moradkhani et. al\cite{Moradkhani2005} extends the Ensemble Kalman Filter into a dual state configuration and successfully predicts a set of parameters.
	
	In this paper hierarchical modeling techniques are combined with a Dual State Ensemble Kalman Filter's parameter perturbation equation to create a Hierarchical Dual State Ensemble Kalman Filter. A hierarchical parameter perturbation framework allows the model to account for parameters that are spatially similar. To examine the Hierarchical Dual State Ensemble Kalman Filter's application to high dimensional geospatially distributed raster data Professor Marko Maneta's \textbf{daWUAPhydroengine}, a hydrological rainfall-runoff model that predicts streamflows across the state of Montana, is used. \textbf{daWUAPhydroengine} is informed by a variety of smaller models, including a groundwater model, snow water equivalient (swe) model, and agricultural component model. Each of these smaller models are influenced by a set of uncalibrated high-dimensional parameters that are stored as 2D raster data. Despite this complexity, \textbf{daWUAPhydroengine} is designed to be a quick and efficient model. Accordingly, a Hierarchical Dual State Ensemble Kalman Filter is an optimal calibration algorithm to calibrate this raster data because 1) the HDEnKF does not have to compute the high dimensional state covariance matrix and 2) \textbf{daWUAPhydroengine} is a sequential model.
	
	Chapter 2 covers the methods behind the Hierarchical Dual Ensemble Kalman Filtering algorithm. Chapter 3 discusses the application of a Hierarchical Dual Ensemble Kalman Filter to \textbf{daWUAPhydroengine}. Chapter 4 discusses results and compares those results with calibrated parameters from a Dual State Ensemble Kalman Filter as implemented by Moradkhani et al.
	
	
	

	