%% This is an example first chapter.  You should put chapter/appendix that you
%% write into a separate file, and add a line \include{yourfilename} to
%% main.tex, where `yourfilename.tex' is the name of the chapter/appendix file.
%% You can process specific files by typing their names in at the 
%% \files=
%% prompt when you run the file main.tex through LaTeX.
\chapter{The Hierarchical Dual Ensemble Kalman Filter Method}


\section{General Dynamic Model and Observations}


A generic dynamic model can be defined as one more more discrete nonlinear stochastic processes\cite{Chen2008}:

\begin{equation}\label{eq:gen_stoc}
\mathbf{x_{t+1}} = f(\mathbf{x_{t}}, \mathbf{u_{t}}, \mathbf{\theta_{t}}) + \varepsilon_{\mathbf{t}}
\end{equation}

where $\mathbf{x_{t}}$ is an $n$ dimensional vector representing the state variables of the model at time step $\mathbf{t}$, $\mathbf{u_{t}}$ is a vector of forcing data (e.g temperature or precipitation) at time step $\mathbf{t}$, and $\mathbf{\theta_{t}}$ is a vector of model parameters which may or may not change per time step (e.g soil beta - $\beta$ or degree day factor -  \textit{DDF}). The non-linear function $f$ takes these variables as inputs and returns the updated state vector at the next timestep $\mathbf{x_{t+1}}$. The noise variable $\varepsilon_{\mathbf{t}}$ accounts for both model structural error and for any uncertainty in the forcing data.

A state's observation vector $\mathbf{z_{t}}$ can be defined as

\begin{equation}\label{eq:gen_obs}
\mathbf{z_{t}} = h(\mathbf{x_{t}}, \mathbf{\theta_{t}}) + \delta_{\mathbf{t}}
\end{equation}

Where the $\mathbf{x_{t}}$ vector represents the true state, $\mathbf{\theta_{t}}$ represents the true parameters, $h(.)$ is a function that determines the relationship between observation and state vectors, and $\delta_{\mathbf{t}}$ represents observation error. $\delta_{\mathbf{t}}$ is Gaussian and independent of $\varepsilon_{\mathbf{t}}$.

The Dual Hierarchical State Ensemble Kalman Filter can be split into three subsections: The prediction phase, the parameter correction phase, and the state correction phase. 

\section{DEnHKF Method}

\subsection{Prediction Phase}

Just as in a standard Dual Ensemble Kalman filter \cite{Moradkhani2005}, each ensemble member $\mathbf{i}$ is represented by a stochastic model similar to \eqref{eq:gen_stoc}. The modified equation is as follows:

\begin{equation}\label{eq:hdekf_predict}
\mathbf{x_{t+1}^{i-}} = f(\mathbf{x_{t}^{i+}}, \mathbf{u_{t}^{i}}, \mathbf{\theta^{i-}_{t}}) + \omega_{\mathbf{t}}, \quad \mathbf{i}=1,...,n
\end{equation}

Where $n$ is the total number of ensembles. The $-/+$ superscripts denote filtered ($+$) and uncorrected ($-$) values. Note that $\mathbf{\theta^{i-}_{t}}$'s $\mathbf{t}$ subscript does not necessarily denote that $\theta$ is time dependent when implemented in the standalone model but rather indicates that parameter values change as they are filtered over time. The noise term $\omega_{t}$ accounts for model error and will hereafter be excluded from the state equation since all error is included in forcing data, see below \cite{Evensen1997}.

Errors in the model design and process noise are accounted for through the perturbation the forcing data vector $\mathbf{u_{t}}$ with random noise $\zeta_{\mathbf{t}}^{\mathbf{i}}$ to generate a unique vector $\mathbf{u_{t}^{i}}$ for each ensemble \cite{Moradkhani2005}, \cite{Chen2008}. $\zeta_{t}^{i}$ is drawn from a normal distribution with a covarience matrix $Q_{\mathbf{t}}^{\mathbf{i}}$.

\begin{equation}\label{eq:hdekf_u}
\mathbf{u_{t+1}^{i}} = \mathbf{u_{t}} + \zeta_{t}^{i}, \quad \zeta_{t}^{i} \sim N(0,Q_{t}^{i}) 
\end{equation}

\subsubsection{Parameter Perturbation}

To generate the apriori parameters $\mathbf{\theta^{i-}_{t+1}}$, an evolution of the parameters similar to the evolution of the state variables must be implemented. Implementations of parameter evolution in \cite{TodiniESzollosi-NagyA1976} added a small perturbation sampled from $N(0,\Sigma^{\theta}_{t})$, where $\Sigma^{\theta}_{t}$ represents the covariance matrix of $\theta$ at timestep $t$. This legacy method of evolution resulted in overly dispersed parameter samples and the loss of continuity between two consecutive points in time \cite{Liu2000} \cite{Chen2008}. To overcome this the kernel smoothing technique developed by West \cite{West1993} and implemented by Liu \cite{Liu2000} has been used effectively in previous Dual Ensemble Kalman filter implementations \cite{Moradkhani2005} and similar models \cite{Chen2008}.

\begin{equation}\label{eq:hdekf_thetaminus_old}
\mathbf{\theta_{t+1}^{i-}} = a\mathbf{\theta_{t}^{i+}} + (1-a)\mathbf{\bar{\theta}_{t}^{+}} + \mathbf{\tau_{t}^{i}}
\end{equation}
\begin{equation}\label{eq:hdekf_tau_old}
\mathbf{\tau_{t}^{i}} = N(0, h^{2}V_{t})
\end{equation}

Where $\mathbf{\bar{\theta}_{t}^{+}}$ is the mean of the parameters with respect to the ensembles, $V_{t} = var(\mathbf{\theta_{t}^{i+}})$, $a$ is a shrinkage factor between (0,1) of the kernel location, and $h$ is a smoothing factor. $h$ may be defined as $\sqrt{1-a^{2}}$. In previous research $a$ values chosen between (.45,.49) have been shown to be optimal \cite{Chen2008}, but note that $h$ and $a$ tend to vary per model and optimal values for these parameters are generally found via experimentation  \cite{Moradkhani2005}  \cite{Anderson1999} \cite{Annan2005} \cite{Chen2008}.

\subsubsection{Hierarchical Parameter Perturbation}

In a standard hierarchical linear regression, a value $y$ and its predictor variables $\alpha$ and $\beta$ are contained in vectors $\mathbf{g_{y}}$, $\mathbf{g_{\alpha}}$, and $\mathbf{g_{\beta}}$ respectively, all of which are of size $N, j=1,...,n$. Vectors $g_{\alpha}$ and $g_{\beta}$ have a mean and standard deviation of $\mu_{\alpha}, \sigma_{\alpha}$ and $\mu_{\beta} , \sigma_{\beta}$ respectively.

\begin{equation}\label{eq:hiearch_standard}
y_{j,g} = \alpha_{g} x_{j,g} + \beta_{g}, \quad \alpha_{g} \sim N( \mu_{\alpha}, \sigma_{\alpha} ), \quad \beta_{g} \sim N( \mu_{\beta} , \sigma_{\beta} )
\end{equation}

where $\alpha$ and $\beta$ are determined to be hierarchically related properties drawn from their respective normal distributions. For a simple overview of hierarchical models refer to Osborne \cite{Osborne2000}, while \cite{Gelman2013} is a more in-depth reference.

In a Hierarchical Duel Ensemble Kalman Filter, parameter perturbation has been modified to have properties of a hierarchical linear regressive system. First, members of $\theta$ are sorted into a series of groups in set $G$. A subset of $G$ where all selected members are in the same group is henceforth referred to as $G_g$. Each member $j$ of a group vector $G_g$, where $g$ is the specific group number, is related to other members through shared hierarchical characteristics (spatial or otherwise.) Algorithms \eqref{eq:hdekf_thetaminus_old} and \eqref{eq:hdekf_tau_old} are then updated to conform to the hierarchical structure described in \eqref{eq:hiearch_standard}:

\begin{equation}\label{eq:hdekf_theta}
\mathbf{\theta_{t+1,g}^{i-}} = a \Upsilon_{t,g}^{i} + (1-a)\mathbf{\bar{\theta}_{t,g}^{+}} + \mathbf{\tau_{t,g}^{i}}
\end{equation}

%%\begin{equation}\label{eq:hdekf_theta}
%%\theta_{t+1}^{i-,g} = a \bar{\theta}_{t,g}^{+} + (1-a)\braket{\theta}_{t,g}^{i+} + \tau_{t,g}^{i}
%%\end{equation}

\begin{equation}\label{eq:hdekf_tau}
\Upsilon_{t,g}^{i} = N( \mu_{g}, \sigma_{g} )
\end{equation}

\begin{equation}\label{eq:hdekf_tau}
\mathbf{\tau_{t}^{i}} = N(0, h^{2}V_{t,g}^{i})
\end{equation}

%%\begin{equation}\label{eq:hddekf_V}
%%V_{t}^{i} = a * var(\theta_{t,g}) + (1-a) var(\theta_{t}^{i-})
%%\end{equation}

Where $\mathbf{\bar{\theta}_{t,g}^{i+}}$ is the mean over all ensembles for all members of group $g$, $\mu_{g}$ and $\sigma_{g}$ are the grand mean and grand standard deviation respectively of all ensembles and locations in the set of vectors in $G_g$ (for clarity, the calculation of the grand mean and grand standard deviation returns a scalar value), and $V_{g,t}$ is the variance matrix with respect to the ensembles of all members of group $g$.

The final modification to be made allows for the dynamic calculation of the shrinkage factor $a$. Since $\mathbf{\tau_{t}^{i}}$ is dependent on the standard deviation of the ensemble $\mathbf{\theta_{t,g}^{i+}}$, a group of ensembles that have tightened around a group of parameters will be increasingly be drawn to the grand mean $\mu_{\theta}$, decoupling tight ensembles from their chosen values and ultimately causing all values across a hierarchical group to collapse to one value. To prevent this, a vector $\alpha$ is substituted for $a$ such that

%\begin{equation}\label{eq:hdekf_new_alpha}
%\mathbf{\alpha_{t,g}^{i}} = a * \frac{\mathbf{\sigma_{t,g}}}{s}
%\end{equation}

%where $\mathbf{\sigma_{t,g}}$ is a vector of standard deviations with respect to the ensembles of all members of group $g$ and $s$ represents the largest possible value for $\mathbf{\sigma_{t,g}}$, generally drawn from minimum and maximum boundary parameters for $\theta$. All members of vector $\mathbf{\alpha_{t,g}^{i}}$ should fall between 0 and 1, with $a$ acting as a tuning parameter.

\begin{equation}\label{eq:hdekf_new_alpha}
\mathbf{\alpha_{t,g}^{i}} = a * (\frac{2}{1+e^{-b\mathbf{v}}} - 1)
\end{equation}

where $\mathbf{v}$ is a vector of variances with respect to the ensembles of all members in group $g$ and $b$ is a tuning parameter that controls how quickly $e^{-b\mathbf{v}}$ converges to 0. $b$ must be chosen very carefully and will vary from parameter to parameter, with the simple equation $b = 1/totalrange$ acting as a possible starting point. All members of vector $\mathbf{\alpha_{t,g}^{i}}$ will fall between 0 and $a$, with $a$ being a value < 1.

Thus, the final equation is

\begin{equation}\label{eq:hdekf_theta_final}
\mathbf{\theta_{t+1,g}^{i-}} = \alpha \Upsilon_{t,g}^{i} + (1-\alpha)\mathbf{\bar{\theta}_{t,g}^{+}} + \mathbf{\tau_{t,g}^{i}}
\end{equation}

\subsection{Parameter Correction Phase}

In an Ensemble Kalman Filter, observations are perturbed to reflect model error. To accomplish this $n$ unique perturbations are created. Therefore, the variable $\mathbf{z_{t+1}^{i}}$ is defined as follows:

\begin{equation}\label{eq:hdekf_obs}
\mathbf{z_{t+1}^{i}} = \mathbf{z_{t+1}} + \mathbf{\eta_{t+1}^{i}},\quad \mathbf{\eta_{t+1}^{i}} = N(0,R_{t+1})
\end{equation}

Where $z_{t+1}$ is an observation vector defined by \eqref{eq:gen_obs} and $\eta_{t+1}^{i}$ is a random perturbation drawn from a normal distribution with covarience matrix $R_{t+1}$. A set of state predictions that can be related to the observations are generated by running the apriori state vector through the function $h(.)$:

\begin{equation}\label{eq:hdekf_pred}
\mathbf{\hat{y}_{t+1}^{i}} = h(\mathbf{x_{t+1}^{i-}}, \mathbf{\theta_{t+1}^{i-}})
\end{equation}

The parameter update equation is similar to the update equation of the linear Kalman filter $\mathbf{x^{+}_{t}} = \mathbf{x^{-}_{t}} + K_{t}(\mathbf{z_{t}}-H\mathbf{\hat{x}_{t}})$, with $\mathbf{\hat{y}_{t+1}^{i}}$ serving the same purpose as $H\mathbf{\hat{x}_{t}}$. However, unlike the linear kalman filter,  parameters are corrected in lieu of the states in the first correction phase:

\begin{equation}\label{eq:hdekf_param_update}
\mathbf{\theta_{t+1}^{i+}} = \mathbf{\theta_{t+1}^{i-}} + K_{t+1}^{\theta}(\mathbf{z_{t+1}^{i}}-\mathbf{\hat{y}_{t+1}^{i}})
\end{equation}

To facilitate this, $K_{t+1}^{\theta}$ is defined as

\begin{equation}\label{eq:hdekf_param_k}
K_{t+1}^{\theta} = \Sigma^{\theta,\hat{y}}_{t+1}(\Sigma^{\hat{y},\hat{y}}_{t+1} + R_{t+1})^{-1}
\end{equation}

where $\Sigma^{\theta,\hat{y}}_{t+1}$ is the cross covariance of $\mathbf{\theta_{t+1}}$ and $\mathbf{\hat{y}_{t+1}}$, $\Sigma^{\hat{y},\hat{y}}_{t+1}$ is the covarience of $\mathbf{\hat{y}_{t+1}}$, and $R_{t+1}$ is the observation error matrix from \eqref{eq:hdekf_obs}. 

\subsection{State Correction Phase}

After $\mathbf{\theta_{t+1}^{i+}}$ has been calculated the model is run again \eqref{eq:hdekf_predict} with the $\mathbf{\theta_{t+1}^{i+}}$ replacing $\mathbf{\theta_{t+1}^{i-}}$.

\begin{equation}\label{eq:hdekf_predict_2}
\mathbf{x_{t+1}^{i-}} = f(\mathbf{x_{t}^{i+}}, \mathbf{u_{t}^{i}}, \mathbf{\theta^{i+}_{t}}), \quad i=1,...,n
\end{equation}

After a new state vector is generated it is re-run through \eqref{eq:hdekf_pred} with the new parameter vector:

\begin{equation}\label{eq:hdekf_pred_2}
\mathbf{\hat{y}_{t+1}^{i}} = h(\mathbf{x_{t+1}^{i-}}, \mathbf{\theta_{t+1}^{i+}})
\end{equation}

The corrected state vector is then run through the state update equation

\begin{equation}\label{eq:hdekf_state_update}
\mathbf{x_{t+1}^{i+}} = \mathbf{x_{t+1}^{i-}} + K_{t+1}^{x}(\mathbf{z_{t+1}^{i}}-\mathbf{\hat{y}_{t+1}^{i}})
\end{equation}
 
\begin{equation}\label{eq:hdekf_param_k}
K_{t+1}^{x} = \Sigma^{x,\hat{y}}_{t+1}(\Sigma^{\hat{y},\hat{y}}_{t+1} + R_{t+1})^{-1}
\end{equation}

where $\Sigma^{x,\hat{y}}_{t+1}$ is the cross covariance of $\mathbf{x_{t+1}}$ and $\mathbf{\hat{y}_{t+1}}$.



